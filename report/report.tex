\documentclass{article}

% if you need to pass options to natbib, use, e.g.:
% \PassOptionsToPackage{numbers, compress}{natbib}
% before loading nips_2018

% ready for submission
\usepackage{nips_2017}

% to compile a preprint version, e.g., for submission to arXiv, add
% add the [preprint] option:
% \usepackage[preprint]{nips_2018}

% to compile a camera-ready version, add the [final] option, e.g.:
% \usepackage[final]{nips_2018}

% to avoid loading the natbib package, add option nonatbib:
% \usepackage[nonatbib]{nips_2018}

\usepackage[utf8]{inputenc} % allow utf-8 input
\usepackage[T1]{fontenc}    % use 8-bit T1 fonts
\usepackage{hyperref}       % hyperlinks
\usepackage{url}            % simple URL typesetting
\usepackage{booktabs}       % professional-quality tables
\usepackage{amsfonts}       % blackboard math symbols
\usepackage{nicefrac}       % compact symbols for 1/2, etc.
\usepackage{microtype}      % microtypography

\title{Iterated Communication Through Negotation}

% The \author macro works with any number of authors. There are two
% commands used to separate the names and addresses of multiple
% authors: \And and \AND.
%
% Using \And between authors leaves it to LaTeX to determine where to
% break the lines. Using \AND forces a line break at that point. So,
% if LaTeX puts 3 of 4 authors names on the first line, and the last
% on the second line, try using \AND instead of \And before the third
% author name.

\author{
  Michael~Noukhovitch\\
  MILA\\
  \texttt{michael.noukhovitch@umontreal.ca} \\
  %% examples of more authors
  %% \And
  %% Coauthor \\
  %% Affiliation \\
  %% Address \\
  %% \texttt{email} \\
  %% \AND
  %% Coauthor \\
  %% Affiliation \\
  %% Address \\
  %% \texttt{email} \\
  %% \And
  %% Coauthor \\
  %% Affiliation \\
  %% Address \\
  %% \texttt{email} \\
  %% \And
  %% Coauthor \\
  %% Affiliation \\
  %% Address \\
  %% \texttt{email} \\
}

\begin{document}

\maketitle

\begin{abstract}
  The abstract paragraph should be indented \nicefrac{1}{2}~inch
  (3~picas) on both the left- and right-hand margins. Use 10~point
  type, with a vertical spacing (leading) of 11~points.  The word
  \textbf{Abstract} must be centered, bold, and in point size 12. Two
  line spaces precede the abstract. The abstract must be limited to
  one paragraph.
\end{abstract}

\section{Introduction}

One of the first philosophers of language, Ludvig Wittgenstein, posited that
"language is use" \cite{wittgenstein2009philosophical}. This idea, that the use
of language is what gives it its meaning, is a profound statement that also has
consequences for how we think of language. Wittgenstein saw language as wholly
tied to its use, there could be no language separate from reality or possible
use. To this end, he defined language games as games with simpler forms of
language "consisting of language and the actions into which it is woven".

Recently, the AI community has taken this philosophy of language and sought to
use it as the basis for the communication of autonomous agents
\cite{wagner2003progress}. The field of "emergent communication" seeks to
understand language starting from the most basic of language games; the goal is
to teach agents to communicate amongst themselves grounded in a simpler world
described by some "game." This game can be one of

\section{Related Work}%
\label{sec:related_work}

\section{Reproduction}%
\label{sec:reproduction}

\subsection{Emergent Communication Through Negotation}%
\label{sub:emergent_communication_through_negotation}

\subsection{Utility }%
\label{sub:utility_}



\section{Exploratory Extensions}%
\label{sec:exploratory_experiments}

\subsection{Utility Sampling}%
\label{sub:utility_sampling}
A negotiation game is not interesting if the interests of the two parties do not
clash. One consequence of randomly sampling utilities is that there is no
guarantee on the clash of utilities in negotation as the players could have
non-zero utility only for the items that their opponent has zero utility and
negotiation is simplified. To combat this issue, it is proposed to guaranteee
non-zero utility to every item.

Another issue is that in the social case, one player's utilities could dominate
the other's $u_j^1 > u_j^2 \forall j$. In such a case, the optimal strategy for
both players is to give all items to the player with the dominating utility, and
again the player. The final split is therefore pareto optimal, but doesn't feel
"fair" for the side of the dominated player. A similar situation for a selfish
agent would generally lead to a more even split with a smaller total reward. For
this reason, we can experiment with avoiding domination situations by
normalizing the utilities so that each agents utilities all sum to 15.

\section{Iterative Negotation}%
\label{sec:iterative_negotation}

\subsection{Pareto Optimality}%
\label{sub:pareto_optimality}

\section{Conclusion}%
\label{sec:conclusion}

\subsubsection*{Acknowledgments}

Use unnumbered third level headings for the acknowledgments. All
acknowledgments go at the end of the paper. Do not include
acknowledgments in the anonymized submission, only in the final paper.

\section*{References}

\bibliography{report}
\bibliographystyle{plain}

\end{document}
